\documentclass[11pt,fleqn,twoside]{article}
\usepackage{makeidx}
\makeindex
\usepackage{palatino} %or {times} etc
\usepackage{plain} %bibliography style 
\usepackage{amsmath} %math fonts - just in case
\usepackage{amsfonts} %math fonts
\usepackage{amssymb} %math fonts
\usepackage{lastpage} %for footer page numbers
\usepackage{fancyhdr} %header and footer package
\usepackage{mmpv2} 
\usepackage{hyperref}

\usepackage{enumitem}

% the following packages are used for citations - You only need to include one. 
%
% Use the cite package if you are using the numeric style (e.g. IEEEannot). 
% Use the natbib package if you are using the author-date style (e.g. authordate2annot). 
% Only use one of these and comment out the other one. 
\usepackage{cite}
%\usepackage{natbib}

\begin{document}

\name{Benjamin Brooks}
\userid{beb12}
\projecttitle{Aberystwyth Web Evaluation \\ Surveys Of Module Experiences \\ \textit{(AWESOME)}}
\projecttitlememoir{\textit{AWESOME}} %same as the project title or abridged version for page header
\reporttitle{Outline Project Specification}
\version{0.1}
\docstatus{Draft}
\modulecode{CS39440}
\degreeschemecode{G401}
\degreeschemename{Computer Science}
\supervisor{Hannah Dee} % e.g. Neil Taylor
\supervisorid{hmd1}
\wordcount{}

%optional - comment out next line to use current date for the document
%\documentdate{6th February 2015} 
\mmp

\setcounter{tocdepth}{3} %set required number of level in table of contents


%==============================================================================
\section{Project description}
%==============================================================================
%  - Main substance of the project
%  - Those aspects that are essential in making it worthwhile
%  - The end goals of the project

The Aberystwyth Web Evaluation Surveys Of Module Experiences (AWESOME), is primarily a tool that enables departments to gather feedback by students about modules, lecturers, and other departmental issues.
It is intended to replace and improve upon the current method of collecting feedback via Google Forms.
This can be achieved by providing a personalised survey for each student to make questions personalised whilst also keeping results anonymous and confidential, and being able to chase up students for not answering.

Advanced analytics and reports is also a feature which is highly requested by university management.
For example, the system needs to be able to extract textual comments from the worst performing modules containing the word `Feedback'.
This can help upper management identify problematic areas in the university and look into the issue further.

AWESOME is a PHP web application developed by Keiron O'Shea during the summer of 2014 under the supervision of Dr.~Hannah Dee.
This project's goal is to bring the current prototype up to a functioning, implementable, and extensible standard.
Security of the system is critical, and so implementing a continuous integration system with unit tests and vulnerability scanning is vital to get up and running early on in the project.
Additionally, the system must be multilingual and accessible to adhere to the university's policies.

%==============================================================================
\section{Proposed tasks}
%==============================================================================
%  - Summarise tasks that will form the major part of work
%    - Lots of research reading
%    - Coding techniques to learn, APIs, etc.
%  - List of tasks you should focus on the major items of work

The project is currently written in procedural PHP, using Twig\cite{Twig} as a templating engine.
In order to make the program more extensible and easier to maintain, it would make sense to refactor the current codebase to follow an object oriented (OOP) model-view-controller (MVC)\cite{PHPMVC} architectural pattern.

The number one priority on this project is to do a security audit on the software currently as it stands and a sanity check on the logic behind it.
After this, the design of the OOP MVC version of the software can begin as well as the other tasks listed.

\subsection{Short Term Tasks}
\begin{description}[itemsep=-0.25em,itemindent=-2em,leftmargin=4em]
	\item[Change procedural design] -- Design a new software architecture using MVC with OOP.
	\item[Unit Testing] -- Use Travis CI\cite{TravisCI} to do automated unit testing and vulnerability scanning.
	\item[Protect admin dashboard] -- Using LDAP HTTP authentication via \emph{.htaccess}.
	\item[PHP Data Objects (PDO)] -- Change the current \emph{mysqli} and \emph{tidy\_sql} implementation to use PDO for greater security, flexibility and features when interacting with databases.
\end{description}

\subsection{Long Term Tasks}
\begin{description}[itemsep=-0.25em,itemindent=-2em,leftmargin=4em]
	\item[Internationalisation (i18n)] -- Reimplement i18n system to support additional languages.
	\item[Accessibility (a11y)] -- Ensure all student-facing pages are accessible for disabled users.
	\item[Relational Database] -- Modify the database schema to be object oriented and relational.
	\item[Analytics/Reports] -- Create a system which can narrow down responses to criteria (e.g. Find modules with a low satisfaction score which mention `feedback' in comments)
\end{description}


\subsection{Future Considerations}
\begin{description}[itemsep=-0.25em,itemindent=-2em,leftmargin=4em]
	\item[Traffic Light Dashboard] -- Have a dashboard showing traffic lights for all modules and departments to detect current issues.
\end{description}



%==============================================================================
\section{Project deliverables}
%==============================================================================
%  - List all key outputs during project
%    - Specified items of working software
%    - Any reviews that you see as fundamental importance to project
%    - Documentation for requirements, design and testing
%    - final report
%  - Explain these items and highlight when they will be produced
%  - Take into account your progress

% TODO: Testing Dates
\begin{description}[itemindent=-2em,leftmargin=4em]
	\item[TODO: Temperature Test] -- 2015-03 -- Internal/closed functional testing.
	\item[TODO: Temperature Questionnaire] -- 2015-03 -- Questionnaire sent out to two departments.
	\item[--]
	\item[Outline Project Specification] -- 2015-02-06 -- This document.
	\item[OOP MVC Class Diagram] -- 2015-02-09 -- UML Class diagram to describe MVC design.
	\item[OOP MVC Release] -- 2015-02-20 -- Functional OOP MVC version of the software.
	\item[i18n and a11y] -- 2015-02-24 -- Add internationalisation and accessibility support.
	\item[Mid-Project Demonstration] -- 2015-03-09 -- Start date of Mid-Project Demonstrations.
	\item[Final Report] -- 2015-05-07 -- Final report hand-in.
	\item[Final Demonstrations] -- 2015-05-11 -- Final project demonstrations.
\end{description}

%==============================================================================
% BIBLIOGRAPHY
%==============================================================================
\nocite{*} % include everything from the bibliography, irrespective of whether it has been referenced.

% the following line is included so that the bibliography is also shown in the table of contents. There is the possibility that this is added to the previous page for the bibliography. To address this, a newline is added so that it appears on the first page for the bibliography. 
%\newpage
\addcontentsline{toc}{section}{Initial Annotated Bibliography} 

% example of including an annotated bibliography. The current style is an author date one. If you want to change, comment out the line and uncomment the subsequent line. You should also modify the packages included at the top (see the notes earlier in the file) and then trash your aux files and re-run. 
%\bibliographystyle{authordate2annot}
\bibliographystyle{IEEEannot}
\renewcommand{\refname}{Annotated Bibliography}  % if you put text into the final {} on this line, you will get an extra title, e.g. References. This isn't necessary for the outline project specification. 
\bibliography{mmp} % References file

\end{document}
