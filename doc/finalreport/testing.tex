\chapter{Testing}
	
	\section{Automated Testing}
	
	For this project, a \ac{CI} platform was chosen to be used in order to facilitate automated unit tests.
	By using \ac{CI} ensured that every time code was committed with unit tests, the suite was ran and results were instantly available.
	Any test failures resulted in a notification email being sent to identify when and where a problem had occurred.
	
	After seeing fellow classmates use TravisCI\footnote{TravisCI Homepage: \url{http://travis-ci.com}} on previous projects, and reviewing the features it provides, I decided to utilise it in this project.
	Travis offered testing across multiple PHP versions, which proved helpful when trying to provision a server on the university network.
	
	This was indispensable mid-way through the project, when functionality in the \ac{i18n} framework needed to be changed and tests started to fail.
	I wouldn't have noticed any errors when manually testing, but the unit tests brought up edge cases which were then dealt with.
	
	\section{User Testing}
	
	User testing was carried out by creating a survey and entering volunteer's \ac{AU} usernames in the student \ac{CSV} data along with sample modules.
	The first real test through the \ac{AU} servers was sent to staff.
	This uncovered many issues with the setup of \ac{AWESOME} on the \ac{AU} server, which took some time to fix.
	After these issues were resolved, an end-of-semester questionnaire was sent out to the vast majority of students in the Computer Science department.
	From first years, to masters students.
	
	User testing revealed some useful information via the feedback form, as well as through a question in the survey which was asked about how easy \ac{AWESOME} was to use.
	
	\section{Acceptance Testing}
	
	Acceptance testing was carried out on the submitted version of \ac{AWESOME} and results in test tables can be found in Appendix \autoref{app:testtables} with 16/18 (89\%) of tests passing.
	More detail of the two failing tests can be found in the appendix entry.
