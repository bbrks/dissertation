\chapter{Background \& Objectives}

	\section{Introduction}
	\label{sec:background}
	
	\acf{AWESOME} is a web-based tool that enables departments to generate personalised questionnaires that get sent out to students to gather feedback about modules, lecturers, or even events such as BCS Show \& Tell\footnote{BCS Mid Wales Show \& Tell: \url{http://midwales.bcs.org/show-and-tell-events}} and other talks.
	
	\section{Background}

	Departments at Aberystwyth currently have no formalised process of gathering student feedback unlike many other universities.
	The University of Sussex requires courses to be evaluated through their \ac{MEQ} \cite{meq} which contains seven core quantitative questions and up to ten additional questions at the school-level, module-level or a mixture of both.
	The University of Westminster have \ac{SME} \cite{sme}, which is an online questionnaire containing ten questions per module sent via e-mail about modules.
	There are many other similar examples of module evaluation systems across UK universities, with their main features being personalised, anonymised, and incentivising completion.
	
	At Aberystwyth, some departments such as Geography \& Earth Science, and English choose to hand out paper-based questionnaires in lectures for each module a student does.
	Having a student take multiple questionnaires to provide essential feedback can lead to many issues due to survey fatigue.
	This can significantly reduce response rates and also have an impact on the answers students provide.
	The Computer Science department has tried many methods of collecting module feedback with varying successes which will be elaborated upon in \autoref{ssec:moduleevaluation}.

	\ac{AWESOME} was originally proposed and developed under the Learning and Teaching Enhancement Fund by Dr. Hannah Dee, and work on the prototype was undertaken by Keiron O'Shea.
	The project was selected as my dissertation project to extend and implement.
	
	Several meetings by the Learning and Teaching Enhancement Committee and Pro-Vice-Chancellor Professor J. Grattan took place to discuss the future of \ac{AWESOME} and the need for a module evaluation system used university-wide and talks are still ongoing.
	
	\subsection{Module Evaluation Method Analysis}
	\label{ssec:moduleevaluation}

	There are several important factors to consider when collecting module evaluation feedback.
	First and foremost, responses must be anonymised, secondly being able to provide incentives for completion drastically increases response rates.
	Being able to have students complete the survey in their own time, and also send targeted reminders to complete the survey are also important.
	If reminders are constantly being sent to students who have already completed the survey, they are likely to get frustrated or annoyed and are less likely to notice another survey e-mail.
	Consolidated surveys help with fatigue, as students only have to answer one survey.
	There have been several studies on how survey fatigue affects response rates and poor answer quality \cite{citeulike:13579648}.
	This consolidation only works if the surveys are personalised, as seen from Google Forms response rates in \autoref{tbl:moduleevaluationmethods}, students are lazy, and they will not skip over modules if asked to.
	
	\autoref{tbl:moduleevaluationmethods} shows a feature comparison of current methods of module evaluation at Aberystwyth University and corresponding response rates.
	
	Paper based module feedback during lecture time can have effective response rates because lecture-time is set aside to complete questionnaires.
	However students usually have to complete one questionnaire per module which can lead to fatigue very quickly.

	Qwizdom\footnote{Qwizdom Homepage: \url{http://qwizdom.com/higher-education/home}} is a hardware-based voting system, with 'clickers' handed to students in a lecture who can then cast their votes through a powerpoint style questionnaire.
	Response rates for Qwizdom module evaluations are high for the same reasons as paper based forms.
	Students are stuck in a lecture for an hour with nothing else to do.
	One problem Qwizdom does give, is that answers can only be quantitative and not qualitative.
	Students can't easily input textual comments through Qwizdom so a lot of valuable information is not gathered from students.

	Google Forms has been the standard way of running module evaluation questionnaires for the past few years in CompSci.
	Google Forms provides anonymous answering in the student's own time and can also provide a way of gathering valuable textual comments.
	One disadvantage of using Google Forms is that there is no way of knowing which modules a student is enrolled for, so students have to skip over modules that aren't applicable.
	This makes the survey confusing and error prone at times and response rates suffer as a result.
		
	\ac{AWESOME} has been created from the ground-up to address all of these problems.
	Responses are anonymised, while retaining the ability to see who has, and has not completed the questionnaire yet.
	This allows for targeted reminders and incentives for completing the survey.
	Additionally, \ac{AWESOME} imports data directly from \ac{ASTRA} which allows all student, staff, and module data to be easily used without lots of manual data entry.
	This also allows for personalised surveys, asking questions only relevant to modules a particular student is enrolled for.
	By collecting both Quantitive and Qualitative data, \ac{AWESOME} can run advanced analytics can be run on the data gathered.
	The questionnaires sent out are also fully responsive, working on phones to tablets, and to desktop computers by utilising Bootstrap\footnotetext{Bootstrap Homepage: \url{http://getbootstrap.com}} and can be completed at any time by the student.
	
	\begin{table}[H]
		
		\vspace{1em}
		
		\begin{tabular*}{\textwidth}{r|ccccccccc|l}
			\hspace{4em} Method & \rot{Tailored Questions} & \rot{Anonymous} & \rot{Qualitative} & \rot{Quantitative} & \rot{Incentives for completion} & \rot{Completion on own time} & \rot{Targeted reminders} & \rot{Responsive} & \rot{Consolidated} & \rot{Response Rate} \\
			\midrule
			Paper         & \KO & \dag & \OK & \ddag & \OK & \KO & -   & -   & \KO & 75\%\cite{qwizdom-student-comprehension} \\
			Qwizdom       & \KO & \OK  & \KO & \OK   & \OK & \KO & -   & \OK & \KO & 50\%\cite{qwizdom-student-comprehension} \\
			Google Forms  & \KO & \OK  & \OK & \OK   & \KO & \OK & \KO & \OK & \OK & 20\% \\
			\acs{AWESOME} & \OK & \OK  & \OK & \OK   & \OK & \OK & \OK & \OK & \OK & TBC  \\
			\bottomrule
		\end{tabular*}
		
		\vspace{1em}
		
		\dag~Anonymity may be compromised when completing paper-based form.
		
		\vspace{1em}
		
		\caption{Module Evaluation methods at Aberystwyth University}
		\label{tbl:moduleevaluationmethods}
		
		\ddag~Manual processing is required in order to analyse the data.
	\end{table}
	
	\section{Objectives}
	\label{sec:projectobjectives}
	
	The overall objective of the project is to implement \ac{AWESOME} on the Aberystwyth University network and collect module feedback for the Computer Science department.
	
	This can be broken down into four main aims:
	
	\begin{itemize}
		\item \textbf{Security Audit the \ac{AWESOME} prototype.} This was a large issue, as there were known security flaws with the prototype and it needed to be looked at immediately before any other work took place.
		\item \textbf{Bring the prototype up to modern development standards.} The program was known to be written in a procedural style, and the security audit brought up the poor \ac{i18n} implementation too.
		\item \textbf{Finish any incomplete functionality.} Many areas of the prototype were half-implemented, but not fully completed. These had to be done before it was useable by students and staff.
		\item \textbf{Run \ac{AWESOME} on a departmental server.} The main objective was to get a survey sent out to students and collect real-world data. This is the final step in that process.
	\end{itemize}
